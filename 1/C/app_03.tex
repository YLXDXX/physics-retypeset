\chapter{常用电磁学量的国际制单位}

电磁学的单位制是一个比较复杂的问题,长期以来存在
着多种单位制,本书采用的是国际单位制.在国际单位制中,
所有的电磁学量,都是由长度、质量、时间、电流强度这四个基
本量导出的.因此,米、千克、秒、安培是电磁学里的基本单位.
下表列出了常用的电磁学量的国际制单位.

\begin{center}
    \begin{tabular}{cc|cc|c}
  \hline
\multicolumn{2}{c|}{物理量} & \multicolumn{2}{c|}{单位} & 量纲式\\
名称 & 符号 & 名称 & 国际符号 \\
  \hline
电流   &  $I$ & 安培 & A & $[I]$\\
电量    &  $Q$  &  库仑  & C   & $[TI]$   \\
电场强度    &$E$    &  伏特每米  & V/m   &  $[LMT^{-3}I^{-1}]$  \\
电势、电势差、电压& $U$ ($V$)   &  伏特  &  V  &  $[L^2MT^{-3}I^{-1}]$  \\
电容    &  $C$  &  法拉  & F   &  $[L^{-2}M^{-1}T^{4}I^{2}]$   \\
电阻    & $R$  & 欧姆  &  $\Omega$ & $[L^2MT^{-3}I^{-2}]$  \\
电阻率    & $\rho$  & 欧姆米  &  $\Omega\cdot {\rm m}$ & $[L^3MT^{-3}I^{-2}]$  \\
磁感应强度    & $B$  & 特斯拉  & T  &  $[MT^{-2}I^{-1}]$ \\
磁通量    & $\phi$  & 韦伯  & Wb  & $[L^2MT^{-2}I^{-1}]$  \\
电感    & $L$  & 亨利  & H  &  $[L^2MT^{-2}I^{-2}]$ \\
  \hline      
    \end{tabular}
\end{center}

\chapter{常用的物理恒量}
\begin{center}
    \begin{tabular}{ll}
        万有引力恒量&    $G=6.67\x10^{-11}\; {\rm N}\cdot {\rm m^2}/{\rm kg}^2$\\
        摩尔气体恒量&    $R=8.31\;  {\rm J}/({\rm mol}\cdot {\rm K})$\\
        阿伏伽德罗常数&    $N=6.02\x 10^{23}\; {\rm mol}^{-1}$\\
        静电力恒量&    $k=9.0\x10^9\;  {\rm N}\cdot {\rm m^2}/{\rm C}$\\
        法拉第恒量&    $F=9.65\x10^4\;  {\rm C}/{\rm mol}$\\
        基本电荷&    $e=1.60\x10^{-19}\;  {\rm C}$\\
        电子的质量&    $m_e=0.91\x10^{-30}\; {\rm kg}$\\
        质子的质量&    $m_p=1.67\x10^{-27}\;   {\rm kg}$\\
        中子的质量&    $m_n=1.67\x10^{-27} \;   {\rm kg}$\\
        $\alpha$粒子的质量&    $m_{\alpha}=6.64\x10^{-27} \;  {\rm kg}$\\
        原子质量单位&    $1{\rm u}=1.66\x10^{-27} \;   {\rm kg}$\\
        真空中光速&    $c=3.00\x10^8\; \ms$\\
        电子的荷质比&    $e/m=1.76\x10^{11}\; {\rm C}/{\rm kg}$ \\
        氢原子的半径&    $a_0=0.53\x10^{-10}\; {\rm m}$   \\
        普朗克恒量&    $h=6.63\x10^{-34}\;  {\rm J}\cdot {\rm s}$   \\
        里德伯恒量&    $R=1.097\x10^7\;  {\rm m}^{-1}$\\        
    \end{tabular}
\end{center}

\chapter{用于构成十进倍数和分数单位的词头}

\begin{center}
    \begin{tabular}{cccc}
        \hline
所表示的因数&词头名称(中文)&词头名称(英文)&词头符号\\
\hline
$10^{18}$ &艾[可萨]&exa-&E\\
$10^{15}$ &拍[它]&peta-&P\\
$10^{12}$ &太[拉]&tera-&T\\
$10^9$ &吉[咖] &giga-&G\\
$10^6$ &兆 &mega-&M\\
$10^3$ & 千 &kilo-& k\\
$10^2$ & 百 &hecto-&h\\
$10^1$ & 十 &deca-&da\\
$10^{-1}$ & 分 &deci-&d\\
$10^{-2}$ &厘 &centi-&c\\
$10^{-3}$ &毫 &milli-&m\\
$10^{-6}$&微&micro-&$\mu$\\
$10^{-9}$&纳[诺]  &nano-&n\\
$10^{-12}$ &皮[可]&pico-&p\\
$10^{-15}$&飞[母托]&femto-&f\\
$10^{-18}$&阿[托]&atto-&a\\
\hline
    \end{tabular}
\end{center}



