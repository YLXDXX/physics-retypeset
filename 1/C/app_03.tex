\chapter{常用电磁学量的国际制单位}

电磁学的单位制是一个比较复杂的问题.
长期以来存在着多种单位制.
本书采用的是国际单位制.在国际单位制中,
所有的电磁学量,都是由长度、质量、时间、电流强度这四个基
本量导出的.因此,米、千克、秒、安培是电磁学里的基本单位.
表~\ref{tab_C_10-1} 列出了常用的电磁学量的国际制单位.

\begin{table}[htbp]
	\centering
	\caption{}\label{tab_C_10-1}
	{
	\zihao{5}
    \begin{tblr}{colspec={cc|ccc|c|c}, cell{1}{1}={c=2}{c}, cell{1}{3}={c=3}{c},cell{1}{6}={r=2}{c},cell{1}{7}={r=2}{c},cells={valign=m},cell{6}{1-Z}={bg=gray!40}}
  		\toprule
		物理量& & 单位 & && 量纲式 & 备注\\
		名称 & 符号 & 名称 &中文符号& 英文符号 & & \\
		  \midrule
		电流   &  $I$ & 安培 &安& $\UA$ & $[I]$ & \\
		电量    &  $Q$  &  库仑  &库& $\UC$   & $[TI]$  &  $1 \text{ 库}=1\text{ 安}\cdot \text{秒}$\\
		电场强度    &$E$    &  伏特每米  &伏/米& $\UVm$   &  $[LMT^{-3}I^{-1}]$  & \\
		{电势、电压\\电势差} & $U(V)$  &  伏特  &伏&  $\UV$  &  $[L^2MT^{-3}I^{-1}]$ & $1 \text{ 伏} =1 \text{ 瓦}/\text{安}$ \\
		电容    &  $C$  &  法拉  &法& $\UF$   &  $[L^{-2}M^{-1}T^{4}I^{2}]$  & $1 \text{ 法}=1 \text{ 库}/\text{伏}$ \\
		电阻    & $R$  & 欧姆  &欧&  $\UO$ & $[L^2MT^{-3}I^{-2}]$  & $1 \text{ 欧}=1 \text{ 伏}/\text{安}$ \\
		电阻率    & $\rho$  & 欧姆米  &欧$\cdot$米&  $\UOm$ & $[L^3MT^{-3}I^{-2}]$  & \\
		磁感应强度    & $B$  & 特斯拉  &特& $\UT$  &  $[MT^{-2}I^{-1}]$ & $1 \text{ 特}=1 \text{ 韦}/\text{米}^2$ \\
		磁通量    & $\phi$  & 韦伯  &韦& $\UWb$  & $[L^2MT^{-2}I^{-1}]$ & $1 \text{ 韦}=1 \text{ 伏}\cdot\text{秒}$ \\
		电感    & $L$  & 亨利  &享& $\UH$  &  $[L^2MT^{-2}I^{-2}]$ & $1 \text{ 享}=1 \text{ 韦}/\text{安}$ \\
		  \bottomrule   
    \end{tblr}
	}
\end{table}

\chapter{常用的物理恒量}
\begin{table}[htbp]
	\centering
	\caption{}\label{tab_C_10-2}
    \begin{tblr}{rl}
        万有引力常量&    $G=6.67\times10^{-11} \UNmqkgq$\\
        摩尔气体恒量&    $R=8.31 \UJmolK $\\
        阿伏伽德罗常数&    $N=6.02\times 10^{23} \Umoln$\\
        静电力恒量&    $k=9.0\times10^9 \UNmqCq $\\
        法拉第恒量&    $F=9.65\times10^4 \UCmol$\\
        基本电荷&    $e=1.60\times10^{-19} \UC$\\
        电子的质量&    $m_e=0.91\times10^{-30} \Ukg$\\
        质子的质量&    $m_p=1.67\times10^{-27} \Ukg$\\
        中子的质量&    $m_n=1.67\times10^{-27} \Ukg$\\
        $\alpha$粒子的质量&    $m_{\alpha}=6.64\times10^{-27} \Ukg$\\
        原子质量单位&    $1{\rm u}=1.66\times10^{-27} \Ukg $\\
        真空中光速&    $c=3.00\times10^8 \Ums$\\
        电子的荷质比&    $e/m=1.76\times10^{11} \UCkg $ \\
        氢原子的半径&    $a_0=0.53\times10^{-10} \Um $   \\
        普朗克恒量&    $h=6.63\times10^{-34} \UJcdots$   \\
        里德伯恒量&    $R=1.097\times10^7 \Umn$\\        
    \end{tblr}
\end{table}

\chapter{用于构成十进倍数和分数单位的词头}

\begin{table}[htbp]
	\centering
	\caption{}\label{tab_C_10-3}
    \begin{tblr}{colspec={cccc},cells={valign=m}}
        \toprule
		所表示的因数&{词头名称\\(中文)}&{词头名称\\(英文)}&词头符号\\
		\midrule
		$10^{18}$ &艾[可萨]&exa-&E\\
		$10^{15}$ &拍[它]&peta-&P\\
		$10^{12}$ &太[拉]&tera-&T\\
		$10^9$ &吉[咖] &giga-&G\\
		$10^6$ &兆 &mega-&M\\
		$10^3$ & 千 &kilo-& k\\
		$10^2$ & 百 &hecto-&h\\
		$10^1$ & 十 &deca-&da\\
		$10^{-1}$ & 分 &deci-&d\\
		$10^{-2}$ &厘 &centi-&c\\
		$10^{-3}$ &毫 &milli-&m\\
		$10^{-6}$&微&micro-&$\mu$\\
		$10^{-9}$&纳[诺]  &nano-&n\\
		$10^{-12}$ &皮[可]&pico-&p\\
		$10^{-15}$&飞[母托]&femto-&f\\
		$10^{-18}$&阿[托]&atto-&a\\
		\bottomrule
    \end{tblr}
\end{table}



