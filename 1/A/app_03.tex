\chapter{国际单位制(SI)}

我们知道,物理公式在确定物理量的数量关系的同时,也
确定了物理量的单位关系.因此,只要我们选定为数不多的
几个物理量的单位,就能够利用它们推导出其他物理量的单
位.这些被任意选定的物理量叫做\textbf{基本量},如力学中的长度、
质量和时间就是三个基本量.
基本量的单位,如米、千克、秒
等,叫做\textbf{基本单位}.
由基本量根据有关公式推导出来的其他
物理量,叫做\textbf{导出量}.
导出量的单位叫做\textbf{导出单位}.


所谓单位制,就是有关基本单位、导出单位等一系列单位
的体制.
由于所采用的基本量的不同,基本单位的不同,以及
用来推导导出单位的定义公式的不同,存在着多种单位制.多
种单位制并用,给科学技术的交流和发展带来不便.
为了避免多种单位制的并存,国际上制订了一种通用的适合一切计
量领域的单位制,叫做国际单位制,国际代号为SI.国际单
位制是1960年第十一届国际计量大会通过的,其后并向全世
界推荐使用.
现在世界上许多国家采用了国际单位制或者正
在向国际单位制过渡,我国也统一实行以国际单位制为基础
的法定计量单位.

在力学范围内,国际单位制规定长度、质量和时间为三个
基本量,它们的单位用米、千克、秒为基本单位.
对于像热学、电磁学、光学等学科,除了上述三个基本单位外,还要加上另
外的基本量,并选定合适的基本单位,才能导出其他物理量的
单位.这样,国际单位制的基本单位共有七个.表~\ref{tab_A_10-4} 和表~\ref{tab_A_10-5} 分别列出了国际单位制的基本单位和常用的力学量的国际单
位制单位.

\begin{table}[htbp]
	\centering
	\caption{国际单位制的基本单位}\label{tab_A_10-4}
	\begin{tblr}{cccc}
		\hline
		\SetCell[r=2]{c} 物理量名称 & \SetCell[r=2]{c} 单位名称 & \SetCell[c=2]{c} 单位符号 \\
		&&中文 & 英文\\
		\hline
		长度 & 米 & 米 & m\\
		质量 & 千克 & 千克 &kg\\
		时间 & 秒 & 秒 &s\\
		电流 & 安培 & 安 &A\\ 
		热力学温度 & 开尔文 & 开 &K\\ 
		发光强度 & 坎德拉 & 坎 & cd\\
		物质的量 & 摩尔 & 摩 & mol\\
		\hline
	\end{tblr}
\end{table}

\begin{table}[htbp]
	\centering
	\caption{常用的力学量的国际单位制单位}\label{tab_A_10-5}
	{
	\zihao{5}
    \begin{tblr}{colspec={cc|ccc|c|c}}
        \hline
         \SetCell[c=2]{c} 物理量     & & \SetCell[c=3]{c} 单位 & & & \SetCell[r=2]{c} 量纲式 &  \SetCell[r=2]{c}备注\\
        名称 & 符号 & 名称 & 中文 & 英文 & & \\
        \hline
        面积    &  $S$    &  平方米    & $\text{米}^2$ & ${\rm m}^2$ & $[L^2]$ & \\
        体积   &  $V$    &   立方米   & $\text{米}^3$ &  ${\rm m}^3$ &$[L^3]$ & \\
        位移   &  $s$    &   米   & $\text{米}$ &  ${\rm m}$ &$[L]$ & \\
        速度   &   $v$   &    米每秒  & $\text{米}/\text{秒}$  & ${\rm m}/{\rm s}$ &$[LT^{-1}]$ &  \\
        加速度   &  $a$    & 米每二次方秒     & $\text{米}/\text{秒}^2$ &  ${\rm m}/{\rm s^2}$ & $[LT^{-2}]$ & \\
        角速度   &  $\omega$    &  弧度每秒    & $\text{弧度}/\text{秒}$ & ${\rm rad}/{\rm s}$ & $[T^{-1}]$ & \\
        角加速度   &  $d$    &   弧度每二次方秒   & $\text{弧度}/\text{秒}^2$ &  ${\rm rad}/{\rm s^2}$ & $[T^{-2}]$ & \\
        转速   &   $n$   &   1每秒   &  $1/\text{秒}$ &  ${\rm s}^{-1}$  & $[T^{-1}]$& \\
        频率   &  $\nu, f$    &  赫兹    & 赫 &  Hz & $[T^{-1}]$ & $1\text{赫}=1\text{秒}^{-1}$ \\
        密度   &  $\rho$    &  千克每立方米    & $\text{千克}/\text{米}^3$ & ${\rm kg}/{\rm m^3}$  &$[L^{-3}M]$ & \\
        力   &  $F$    &  牛顿    &  牛 &  N &$[LMT^{-2}]$ &  {$1\text{牛}=$\\$1 \text{千克}\cdot\text{米}/\text{秒}^2 $} \\
        重量   &  $G$    &    牛顿  &  牛 & N  &$[LMT^{-2}]$ & \\
        力矩   &  $M$    &    牛顿米  & $\text{牛}\cdot \text{米}$ & ${\rm N}\cdot {\rm m}$  & $[L^2MT^{-2}]$ & \\
        动量   &  $p$    & 千克米每秒    & $\text{千克}\cdot \text{米}/\text{秒}$ &  ${\rm kg}\cdot {\Ums}$ &$[LMT^{-1}]$ & \\
        冲量   &  $I$    &  牛顿秒    & $\text{牛}\cdot \text{秒}$ &   ${\rm N}\cdot {\rm s}$&$[LMT^{-1}]$ & \\
        压强   &  $p$    &   帕斯卡   &  帕  &  Pa & $[L^{-1}MT^{-2}]$ & $1 \text{帕} = 1 \text{牛}/\text{米}^2$ \\
        功   &  $W$    &   焦耳   & 焦 &  J & $[L^2MT^{-2}]$ & $1 \text{焦} = 1 \text{牛} \cdot \text{米}$ \\
        能   &   $E$   &   焦耳   & 焦 &  J & $[L^2MT^{-2}]$ & \\
        功率   &  $P$    &  瓦特    &  瓦  & W &$[L^2MT^{-3}]$ & $1\text{瓦}=1\text{焦}/\text{秒}$ \\
        \hline
    \end{tblr}
	}
\end{table}





\chapter{量纲}

我们知道,物理量可分为基本量和导出量.
既然导出量
可以从基本量导出,那么每个导出量一定可以用基本量的某
种组合表示出来.表示一个物理量是由哪些基本量组成和怎
样组成的式子,叫做这个物理量的\textbf{量纲式}.
在国际单位制中,
所有的力学量都是由长度、质量、时间这三个基本量组成的,
如果用$L $,$ M $,$ T$分别表示这三个基本量,那么,一个物理量$Q$
的量纲式的一般形式就是
\[[Q]=\qty[L^{\alpha}M^{\beta}T^{\gamma}]\]
其中$\alpha, \beta, \gamma$分别叫做物理量$Q$对于长度、质量、时间的\textbf{量纲}.

下面举出几个物理量的量纲式:
\begin{itemize}
    \item 体积的量纲式
    \[[V]=\qty[L^3]  \]
    \item 速度的量纲式
    \[[v]=\frac{[s]}{[t]}=\qty[LT^{-1}]  \] 
    \item 加速度的量纲式
    \[ [a]=\frac{[v]}{[t]}=\qty[LT^{-2}] \]
    \item 力的量纲式
    \[ [F]=[m][a]=\qty[LMT^{-2}] \]
\end{itemize}

在上面的表~\ref{tab_A_10-5} 中列出了一些力学量的量纲式.

物理量的量纲式可以用来检验物理关系式的正确性.检
验时所根据的原则是:只有量纲式相同的项才能相加减,而且
等号两边一定要有相同的量纲式.一个正确的物理关系式,
一定是符合上述原则的.






