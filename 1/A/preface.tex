\chapter{说明}

本书是在中小学通用教材物理编写组编的
《全日制十年制学校高中课本(试用本)物理》的基础上,
按照高中物理教学大纲较高要求的内容编写成的.编
写中吸收了几年来各地试用中的一些经验和意见.许多省市的中学教师和有关
高等院校的教师对本书征求意见稿提了有益的意见和建议.北京、安徽、江西、
河南、上海、天津、浙江、江苏、湖北、广东、山西、黑龙江等省市的教研室
和教育学院在本书编写过程中给予了大力支持.在此谨致谢意.

希望广大教师和研究中学物理教学的同志提出批评和修改建议.


\chapter{引言——怎样学好物理知识}

我们在初中学了两年物理,学习了一些物理概念,如质
量、重量、功、能、电流、电压、电阻等等;学习了一些物理定律,
如惯性定律、能的转化和守恒定律、欧姆定律、光的反射定律
等等;初步知道了一些物理理论,如分子论、电子论.这些概
念、定律、理论都是物理知识,正如我们在初中学习物理中体
会到的那样,物理知识是大们认识自然和改造自然的重要
武器.

经过几千年特别是近两三百年的积累,人类的物理知识
已经很丰富了,物理知识的应用已经很广泛了.在初中讲的
只是一些十分浅显的物理知识.为了适应把我国建设成为现
代化的,高度文明、高度民主的社会主义国家的需要,我们在
高中还要进一步学习物理知识.

    在高中,我们要加深对重要的物理知识的理解.例如,初
中讲了力是改变物体运动状态的原因,高中要进一步学习力
是怎样改变物体运动状态的;初中讲了闭合电路的一部分做
切割磁力线的运动时电路中要有感生电流,高中要进一步学
习感生电流的大小是怎样决定的等等.我们在高中还要扩
大物理知识的范围,例如,光到底是什么?常常听说的原子
能、激光等到底是怎么一回事?这些在初中没有讲到的物理
知识在高中都要讲到.在高中我们的物理知识将扩大和加
深.同时,我们学习物理知识的能力以及应用物理知识来分
析解决问题的能力也将得到提高.

    那么,在高中怎样进一步学好物理知识呢?

\subsection*{做好物理实验}
人类的物理知识是怎么得来的呢?想想
看,假使不研究物质的性质随温度的变化,人们能认识物态变
化的规律吗?假使不研究电流使磁针偏转等现象,人们能认
识电流周围存在着磁场吗?假使不研究反射光线和入射光线
的关系,人们能发现光的反射定律吗?整个物理学的发展史告
诉我们,人类的物理知识来源于实践,特别是来源于科学实验
的实践.

    我们学习物理知识的过程,跟人类探索物理知识的过程
有很多相似之处.因此,在高中迸一步学习物理的时候,必须
充分重视实践在学习物理知识中的重要意义,特别是要认真
做好实验.

    实验能够帮助我们形成正确的物理概念,增强观察物理
现象和分析物理问题的能力,加深对物理规律的理解,为了
做好实验,在每次实验之前,一定要明确实验的目的,弄懂它
的原理,了解所用仪器的性能,搞清楚实验的步骤;实验中要
认真观察现象,仔细记录必要的数据;实验后要对所得的数据
进行分析,作出合理的结论,必要时要进一步研究那些还不够
清楚的问题.这里,事先的准备工作特别重要.这是因为,我
们如果事前对实验目的和怎样达到这个目的的步骤都清楚
了,那么,在具体操作中,就能够自觉地有目的地把实验做好.
反之,如果事前不作好必要的准备,实验时只是按照别人拟定
的实验步骤去操作,观察时不知道把注意力集中到重要的现
象上,记录数据时不知道记下这些数据干什么,这样,实验虽
然做过了,收获却是很小的.为了做好实验,并从实验中得到
应有的收获,我们一定要作好事前的堆备,并在整个实验过程
中都要手脑并用.

    对老师的演示实验也要注意观察,并且要在老师的指导
下分析观察到的现象,得出应有的结论.还应努力创造条件
在课外多做一些简易实验.不做实验,不仔细观察物理现象,
是不能学好物理知识的.

\subsection*{学好物理概念和规律}

物理知识来源于实践,但实践的
经验并不就是物理知识.这跟房屋是由砖瓦等建筑材料组成
的,但建筑材料并不就是房屋一样.要把建筑材料变成房屋,
还需要人们进行修建房屋的劳动.与此相似,要从实践经验
中总结出物理知识,人们还必须进行分析、综合等抽象的思维
活动.例如,通过观察和实验,我们发现运动物体受到的阻力
越小,它的速度减小得越慢.但是,只有通过抽象思维,我们
才能得出物体不受外力时将保持匀速直线运动的结论.一般
说,人们在抽象出物理现象的共同属性后,就认识了有关的物
理概念,在抽象出物理现象的变化规律后,就发现了物理规
律.因此,我们必须充分注意在经验事实的基础上是经过怎
样的抽象思维而建立起理论知识的,才能学好物理概念和规
律.

物理概念和规律常常用数学公式来表示.例如密度的公
式$\rho=m/V$,功的公式$W=Fs$,欧姆定律的公式$I=U/R$,电
功率的公式$P=UI$等等,都是我们在初中学过的物理公式,
把概念和规律写成公式后,显得特别简单、明确,而且便于运
用它们来进行分析、推理、论证.物理规律还常常用函数图像
来表示.在初中学过的给晶体和非晶体加热时它们在熔解前
后温度随时间而变化的图线(图 \ref{fig_A_0-1}),就是一个例子.从图
中很容易看出晶体和非晶体在熔解时表现出不同的特点,晶
体在熔解时虽然继续受热但温度并不升高,非晶体就没有这
个特点.图像的突出优点是它很直观,而且比较容易直接根
据实验数据画出来.

\begin{figure}[htbp]
    \centering 
    \begin{subfigure}{0.46\linewidth}
        \centering
        \begin{tikzpicture}[>=stealth, thick]
            \draw [<->](0,3)--(0,0)--(4,0);
            \node at (.5,3) {温度};
            \node at (3.75,-.25) {时间};
            \draw (.5,.5)--(1.5,1.5)--(2.5,1.5)--(3.5,2.5);
            %\node at (2,-.75) {甲:晶体};
        \end{tikzpicture}
        \caption{晶体}\label{fig_A_0-1a}
    \end{subfigure}
    \hfil
    \begin{subfigure}{0.46\linewidth}
        \centering
        \begin{tikzpicture}[>=stealth, thick]
            \draw [<->](0,3)--(0,0)--(4,0);
            \node at (.5,3) {温度};
            \node at (3.75,-.25) {时间};
            \draw (.5,.5) [bend left =15] to (3.5,2.5);
            %\node at (2,-.75) {乙:非晶体};
        \end{tikzpicture}
        \caption{非晶体}\label{fig_A_0-1b}
    \end{subfigure}
    \caption{物体受热时温度随时间而变化的图线}\label{fig_A_0-1}
\end{figure}

    数学知识在物理学中的应用是十分重要的,但是我们却
不可以只从数学的角度来看待物理问题.对于物理概念,要
特别注意它的物理意义,对于物理规律,要特别注意它的适
用范围.

    有人总是认为水越深浮力就越大.其实,水对物体的浮
力等于物体排开的水的重量,跟物体在水中的深度没有关系,
发生这个错误的原因就在于对浮力的物理意义不清楚.有的
初学者从公式$R=U/I$得出结论说,导体的电阻跟加在导体
上的电压成正比,跟通过导体的电流强度成反比.实际上,电
阻是导体本身的属性,是由导体的长度、横截面积和材料决定
的,跟加在导体上的电压和通过导体的电流强度无关.这个
错误就是由于没有搞清楚电阻的物理意义,错误地理解公式
造成的.这些例子以及同学们自己也可以举出的类似的例子
说明,理解概念的物理意义是多么重要.

    物理规律一般都有一定的适用范围.例如,弹簧的伸长
只有在一定的限度内才跟所受的拉力成正比,超出这个限度,
伸长就不跟拉力成正比了,欧姆定律对金属导体是正确的,
对液体导电也适用,对气体导电就不成立了.物理规律不能
随意应用到它的适用范围之外去.例如,机械能在只有动能
和势能发生相互转化时才是守恒的.飞机起飞、火车制动、大
炮发射炮弹的时候,机械能都跟其他形式的能发生相互转化、
这时总的能量是守恒的,但机械能并不守恒.在这些情况下
就不能用机械能守恒定律来分析讨论问题,否则就会得到错
误的甚至荒谬的结果.所以,学习物理规律时,知道它的适用
范围是非常重要的.

    在高中,我们对物理概念和规律的讨论,要比在初中深入
得多,应用物理概念和规律来进行分析、推理和论证的机会也
很多.我们必须注意掌握物理概念和规律的物理意义和适用
范围,这样才能学懂学好物理知识.如果忽视这一要点,只去
死背定义,硬记公式,那是不可能学好物理知识的.


\subsection*{做好练习}

做练习是学好物理知识的必不可少的环节.
认真做好练习,可以加深对所学知识的理解,发现自己知识中
的薄弱环节而去有意识地加强它,逐步培养自己的分析解决
问题的能力,逐步树立解决实际问题的自信心.

物理练习有多种形式,如问答题、实验题、计算题等.怎样才能做好练习,不能一概而论,这里只能初步说明做好物理
练习一般需要注意的几个问题,作为入门的指引.

    我们知道,要处理好一件事情,首先是要把情况摸清楚.
做练习也是这样,先要仔细审题,弄清楚题中叙述的物理过
程.譬如有一道关于物体做机械运动的题,就要先弄清楚物
体是做匀速运动还是变速运动,它原来是静止的,还是本来就
在运动,运动轨迹是直线还是曲线,等等.把物理过程弄清楚
以后,要进一步明确哪些条件是已经知道的,什么是要解决的
问题即所求的答案.这样我们才有一个可靠的出发点.

    在弄清题意之后,再根据题中叙述的物理过程、已知条件
和所求答案来确定应该运用哪些物理规律.这是做好练习的
十分重要的而又往往被初学者忽视的一步.只有经过认真的
分析思考,把应该运用的物理规律找准了,我们才能有把握地
解决问题.否则就可能流于乱套公式,做对了不知道是怎么
对的,做错了也不知道是怎么错的.这样,即使做了很多练
习,也是收不到应有效果的.

    在找出了应该运用的物理规律之后,最后的工作就是利
用这些规律来建立已知条件和所求答案之间的关系,从而求
出答案.这个关系有时比较简单,容易看出来,有时比较复
杂,要逐步去寻找.对于比较复杂的问题怎样去逐步找出已
知条件和所求答案的关系,我们将在以后各章中结合例题来
具体说明.对得到的答案,还应该根据实际情况考虑它是否
合理.譬如所得答案是一个人有几吨重,飞机的速度只有几
厘米每秒,这显然是不合理的.如果发生这种情况,就要认真
检查什么地方出了错.

    做好练习的目的,是为了掌握所学的知识,培养我们运用
所学知识分析和解决问题的能力.希望同学们在做练习中,
既要肯于动脑筋,又要善于动脑筋,这样才能把物理知识真正
学到手,并培养起我们的能力来.不动脑筋,乱套公式,死记
类型,机械模仿,都不能达到做好练习的目的.为了掌握知
识,需要做一定数量的练习,但是,如果误以为学物理就是做
题,既不复习老师讲课的内容,也不阅读教材,就盲目地去找
过多的难题来做,同样不能达到做好练习的目的.
这些错误办法无助于我们学好本领,增长才干,一定要坚决摒弃.



